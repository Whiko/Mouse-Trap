  \chapter*{Rappel du sujet}
   \setcounter{chapter}{1}
    \addcontentsline{toc}{part}{Rappel du sujet}
    \lhead{Rappel du sujet}
    
    \section{Règles de fonctionnement}
      Pour notre projet, nous avons choisi le sujet n°2 : le jeu Mouse-Trap. \\
      Nous rappelons le sujet : ``Le joueur dirige une souris dans un labyrinthe dans le but de manger tous les morceaux de fromage du niveau. Ses ennemis sont des chats et des faucons. La souris 
      peut collecter des os qui lui permettent, à tout moment, de se transformer en chien quelques instants et de chasser les chats. Le joueur dispose de quatre boutons en plus du joystick 
      directionnel. Le premier bouton permet au joueur de se transformer en chien s’il possède, au moins, un os. Les trois autres permettent au joueur de fermer ou d’ouvrir, à tout moment, 
      les portes jaunes, rouges et bleues et ainsi de bloquer ou emprisonner les chats en transformant l’apparence du labyrinthe.``\\
      Pour notre projet, nous nous sommes inspirées du célèbre jeu \texttt{PacMan}. En ce qui concerne les règles retenues, nous voulions respecter toutes les règles de base du jeu Mouse-Trap tout
      en gardant l'esprit ''PacMan``, puis au fur et à mesure nous avons préféré nous focaliser sur les points importants du jeu (tels que les déplacements des personnages ou la possibilité
      de jouer à plusieurs en réseau), pour qu'ensuite potentiellement, une fois fini, nous apportions des améliorations.
      Cependant, nous n'avons pas eu la possibilité de faire ces modifications avant la présentation du projet (nous pensons personnellement à nous y intéresser après les examens finaux).\\
      Les principales fonctionnalités du jeu sont la possibilité de jouer à plusieurs niveaux en mode solo (jusqu'à cinq niveaux), ainsi que de jouer à plusieurs personnes en réseau.
      En mode solo, le joueur controle PacMan et doit ramasser toutes les pièces se trouvant au sol tout en évitant de se faire attraper par les fantomes controlés par l'ordinateur. Lorsque 
      toutes les pièces sont ramassées, le joueur change de niveau. PacMan a également la possibilité de devenir invincible en ramassant l'étoile rouge qui apparait de temps en temps sur la carte, 
      et peut ainsi tuer les fantomes qu'il touche. 
      Lorsque la partie est finie (donc soit que le joueur n'a plus de vie, soit qu'il a réussi à terminer tous les niveaux), le joueur peut entrer son pseudo afin que son score soit
      enregistré.\\
      En mode multijoueur, le premier joueur connecté controle PacMan, les joueurs suivants controlent les fantomes. PacMan a pour objectif de récupérer toutes les pièces du niveau, les fantomes 
      doivent l'en empecher.\\
      Le menu Options propose d'activer ou de désactiver le son, donne les touches utilisées dans le jeu, et également accès aux meilleurs scores réalisés.\\
      
      Les touches utilisées pour controler les personnages sont les fleches directionnelles.
    
    \section{Choix des structures de données}
      Nous voulions pouvoir travailler chacune sur une partie du code simultanément et pouvoir ensuite complémenter notre code, c'est pourquoi nous avons choisi d'utiliser un gestionnaire 
      de versions. Nous avons choisi \texttt{Git} dont nous avions déjà entendu parlé et qui est plutot simple et pratique à utiliser.\\
      Ensuite, lorsque nous avons commencé à réfléchir à notre projet, il nous a fallu trouver une bibliothèque graphique. Nous avons fait quelques recherches : nous avons hésité entre la bibliothèque
      graphique de base avec Java (AWT) et Slick2D. Notre choix s'est finalement porté sur Slick2D, qui est une bibliothèque complète et simple 
      d'utilisation (et qui avait l'air assez conseillée pour faire des jeux en 2D). \\
      
      Ensuite, dans le code, nous voulions pouvoir modifier les constantes sans problème, c'est pourquoi nous avons créé des fichiers de 
      configuration dans lesquels se trouvent les constantes (hauteur et largeur de la carte, vitesse des personnages, taille des images du mur et
      des personnages, positions de départ des fantomes, écart entre le bord de la fenêtre et les bords de la carte etc.). Chaque niveau du mode solo a son propre
      fichier de configuration.
	